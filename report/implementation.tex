\قسمت{زبان برنامه‌نویسی}
برای بازسازی الگوریتم‌های استفاده شده از زبان برنامه نویسی پایتون نسخهٔ ۳٫۶ استفاده شد.

\قسمت{کتابخانه‌ها}
کتابخانه‌های پایتون به کار رفته عبارت‌اند از:
\شروع{فقرات}
\فقره \مل{MOABB}\پانویس{Master of All BCI Benchmarks}\مرجع{Jayaram2018}: از این کتابخانه برای تهیه بنچمارک عملکرد الگوریتم‌ها استفاده شد.
\فقره \مل{Numpy}: از این کتابخانه برای عملیات برداری استاندارد استفاده شد.
\فقره \مل{Scipy}: از این کتابخانه برای اعمال فیلترهای میان‌گذر و تبدیل هیلبرت استفاده شد.
\فقره \مل{Sklearn}: از این کتابخانه برای اعمال عملیات \مل{ICA} و \مل{Feature Selection} استفاده شد. همچنین، از کلاس انتزاعی \ورب{BaseEstimator} این کتابخانه به عنوان کلاس پایهٔ مراحل پیش‌پردازش و پردازش استفاده شد.
\فقره \مل{MNE}\مرجع{EricLarson2020}: از این کتابخانه برای تبدیل \مل{CSP} استفاده شد.
\فقره \مل{Keras}: از این کتابخانه برای تعریف معماری شبکه‌های عصبی کانولوشنال استفاده شد.
\فقره \مل{TensorFlow} از این کتابخانه به عنوان هستهٔ پردازشی کتابخانهٔ \مل{Keras} استفاده شد.
\پایان{فقرات}

\قسمت{الگوها}
تمام مراحل پردازش و پیش‌پردازش در قالب کلاس‌هایی که از پایهٔ \ورب{BaseEstimator} کتابخانهٔ \مل{Sklearn} مشتق شده‌اند، نوشته شد. سپس، این مراحل با استفاده از کلاس \ورب{Pipeline} کتابخانهٔ مذکور، به یکدیگر متصل شدند. آبجکت \ورب{Pipeline} به دست آمده، در کنار دیتاست‌ها و پارادایم‌های مربوطه به توابع \مل{Cross-Validation} کتابخانهٔ \مل{MOABB} داده شدند و بدین ترتیب وظیفهٔ ارزیابی و مقایسهٔ روش‌های مختلف به این کتابخانه سپرده منتقل شد.

\قسمت{دیتاست‌ها}
برای بازسازی پژوهش \آ، به طور دقیق از دیتاست عمومی مورد استفاده در آن مقاله (\مل{BCI Competition IV-2a}) استفاده شد. کتابخانهٔ \مل{MOABB} متدهای لازم برای دانلود این دیتاست و باز کردن آن با فرمت پایتون را در بر دارد.

در مورد پژوهش \ب، دیتاست به کار رفته انحصاری است. بنابراین در این کار نیز از همان دیتاست \آ\ استفاده شد؛ با این تفاوت که از آنجایی که معماری شبکهٔ عصبی خروجی باینری می‌دهد، تنها داده‌های مربوط به حرکت دست راست و حرکت دست چپ مورد استفاده قرار گرفت و داده‌های زبان و پا حذف شد.

برای \پ\ یکی از دیتاست‌ها به صورت عمومی در دسترس است اما از آنجایی که تصمیم بر آن شده بود که هم برای بازسازی \آ\ و هم برای \ب\ از دیتاست \مل{BCI Competition IV-2a} استفاده شود، بنا شد که برای این پژوهش نیز از دیتاست دیگری استفاده نشود. این تصمیم برای سادگی کار و همچنین به وجود آمدن امکان مقایسه بین سه روش گرفته شد.

\قسمت{راهنمای اجرای فایل‌ها}
\ورودی{code-readme}

