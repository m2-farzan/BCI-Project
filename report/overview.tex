\قسمت{مقایسه اهداف مقالات مورد بررسی}
هدف تمام مقات مورد بررسی، طبقه‌بندی امواج مغزی جهت تشخیص دادن حرکت مورد نظر بیمار است. در نهایت، مطلوب است که یک شبکهٔ عصبی به گونه‌ای تمرین داده شود که بتواند با گرفتن ورودی جدید امواج مغزی، مشخص کند که حرکت مورد نظر کاربر چه بوده است. طبقه‌بندی مورد نظر به صورت کلی بوده و خروجی‌هایی مانند «حرکت دست راست» یا «حرکت زبان» مورد انتظار است. به بیان دیگر، استخراج و یا طبقه‌بندی مواردی نظیر «شدت» یا «سرعت» حرکت در این مقالات مورد نظر نمی‌باشد.

تفاوت این مقالات در این است که مقالهٔ \ب\ تلاش کرده‌است مدلی را توسعه دهد که به صورت \مل{Cross-Subject} اعتبار داشته باشد ولی مقاله‌های \آ\ و \پ\ هدفشان را به طبقه‌بندی صحیح در حالت \مل{Single Subject, Cross-Session} محدود کرده‌اند. به بیان دیگر، پژوهش \ب\ قصد دارد شبکهٔ عصبی تولید کند که وزن‌های آن پس از تمرین داده شدن با داده‌های مغزی چند فرد، قابلیت تعمیم به دیگر افراد را نیز داشته باشد اما پژوهش‌های \آ\ و \پ\ فرض کرده‌اند شبکه به ازای هر فرد به طور جداگانه تمرین داده می‌شود.

همچنین، شبکهٔ عصبی مقالهٔ \ب\ برای طبقه‌بندی دوتایی چپ و راست\پانویس{Left-Right Imagery} طراحی شده است اما شبکه‌های عصبی مقالات \آ\ و \پ\ طبقه‌بندی چندتایی انجام می‌دهند؛‌ در مقالهٔ \آ\ خروجی چهار حرکت زبان، پا، دست راست و دست چپ را طبقه‌بندی می‌کند و در مقالهٔ \پ\ خروجی سه حرکت مختلف آرنج که با کدهای زاویه‌ای ۹۰-، ۰ و ۹۰+ برچسب‌گذاری شده‌اند طبقه‌بندی می‌شوند.

\قسمت{مقایسه دیتاست‌های مقالات}
\آ\ از دیتاست \مل{BCI Competition 2008 IV-2a} استفاده می‌کند. این دیتاست به صورت عمومی قابل دسترسی است. در این دیتاست، از ۹ فرد\پانویس{Subject} در ۲ جلسه\پانویس{Session} که در روزهای متفاوتی برگزار شد، استفاده شد. هر جلسه، از ۶ ران\پانویس{Run} تشکیل شد که بین آن‌ها استراحت‌های کوتاهی وجود داشت. در هر ران، ۴۸ بار\پانویس{Trial/Epoch} از کاربر خواسته شد که یکی از ۴ حرکت زبان، پا، دست راست و دست چپ را تصور کند (هر حرکت ۱۲ بار). خروجی نوار مغزی فرد هنگام تصور هر یک از این حرکات در ۲۵ کانال با نرخ نمونه‌برداری ۲۵۰ هرتز ضبط شده و در کنار برچسب حرکت مورد نظر ذخیره شد. جزییات بیشتر درباره تجهیزات نمونه برداری و زمان‌بندی تحریک‌های مختلف در مرجع \مرجع{Tangermann2012} قابل مشاهده است. لازم به ذکر است که در مجموعه \مل{BNCI Horizon 2020} که در این گزارش به کار گرفته شده است، این دیتاست با کد \مل{001-2014} منتشر شده است.

\ب\ از دیتاست انحصاری خود استفاده می‌کند. در این دیتاست، با توجه به رویکرد این مقاله که یافتن الگوهای مشترک بین افراد مختلف است، از تعداد افراد بیشتری استفاده شده است. به طور دقیق، از ۵۴ داوطلب سالم در ۲ جلسه استفاده شد. تعداد ران‌ها و دفعات حرکت در مقاله ذکر نشده است. تنها دو کلاس حرکت مورد بررسی قرار گرفت که شامل حرکت دست راست و چپ بود. نرخ نمونه‌برداری ۱۰۰ هرتز و تعداد کانال ۶۶ عدد بود.

\پ\ نتایج شبکهٔ عصبی طراحی شده را هم با دیتاست انحصاری و هم با یک دیتاست عمومی منتشر کرده است. دیتاست عمومی در مجموعهٔ \مل{BNCI Horizon 2020} با کد \مل{001-2017} در دسترس است. در این دیتاست، از ۱۵ داوطلب سالم در ۲ جلسه استفاده شد. در هر جلسه، ۱۰ ران وجود داشت و در هر ران ۴۲ بار از کاربر خواسته شد یکی از ۶ حرکت فلکشن\پانویس{Flexion} و اکستنشن\پانویس{Extension} آرنج، سوپیناسیون\پانویس{Supination} و پروناسیون\پانویس{Pronation} ساعد و باز و بسته کردن مشت را انجام دهند. نمونه‌برداری با نرخ ۵۱۲ هرتز در ۶۱ کانال \مل{EEG} انجام شد. جزییات بیشتر درباره این دیتاست در مرجع \مرجع{Ofner2017} قابل مشاهده است.

\قسمت{مقایسه پیش پردازش مقالات}
پیش پردازش و نمایش دادهٔ سه مقالهٔ مورد بررسی بسیار متفاوت است.

\زیرقسمت{پیش پردازش \آ}
در مقالهٔ \آ، ابتدا داده‌ها با استفاده از فیلترهای فرکانسی میان‌گذر نوع \مل{Chebyshev II} به فضای فرکانس‌های مختلف برده می‌شود. ۹ باند فرکانسی بدون همپوشانی از ۴ تا ۴۰ هرتز برای این کار انتخاب شده است اما مرتبهٔ فیلتر ذکر نشده است. سپس به ازای هر باند فرکانسی، تحلیل \مل{CSP}\پانویس{Common Spatial Patterns} انجام می‌شود که پایه‌های جدیدی را برای کانال‌های \مل{EEG} انتخاب می‌کند به گونه‌ای که الگوهای مکانی جدا شوند. بدین ترتیب، یک مجموعه فیلتر فرکانسی و یک مجموعه فیلتر فضایی به دست می‌آید.

به ازای هر \مل{Trial}، با محاسبهٔ واریانس سمپل‌های هر سیگنال پس از اعمال زوج فیلترهای فرکانسی و مکانی، یک نقشهٔ دو بعدی به دست می‌آید که معیاری از انرژی سیگنال در آن باند فرکانسی و تبدیل مکانی است.

در مرحلهٔ بعد، این نقشه‌های انرژی در کنار برچسب داده‌ها، به یک الگوریتم \مل{Feature Selection} داده می‌شود تا زوج فیلترهایی که بیشترین اطلاعات را دارند پیدا شوند. نوع الگوریتم \مل{Feature Selection} در مقاله ذکر نشده است اما این مورد آمده است که \مل{Feature Selection} به صورت یک در برابر همه انجام می‌شود و خروجی آن ۳۲ فیچر است. این ۳۲ فیچر هر کدام متناظر با یک زوج فیلتر میان‌گذر فرکانسی و یک ترکیب خطی کانال‌های \مل{EEG} متناظر هستند و پارامتر الگوریتم طبقه‌بندی محسوب می‌شوند که هنگام تمرین مشخص می‌شوند.

با مشخص شدن این زوج فیلترها، پیش پردازش به ازای هر ورودی، سیگنال‌ها را به فضای مورد نظر می‌برد. سپس با استفاده از یک تبدیل هیلبرت\پانویس{Hilbert Transform}، شکل \مل{Envelope} موج به دست می‌آید. از آنجایی که \مل{Envelope} موج فرکانس کمتری از خود موج دارد، سیگنال \مل{downsample} می‌شود تا طول خروجی به ۴۰ نمونه کاهش یابد. بدین ترتیب، به ازای هر \مل{Trial}، الگوریتم پیش‌پردازش ۳۲ فیچر با ۴۰ نمونه زمانی به دست می‌دهد که این ماتریس دو بعدی به عنوان ورودی به شبکه عصبی کانولوشنال خورانده می‌شود.

\زیرقسمت{پیش پردازش \ب}
پیش پردازش این مقاله تا مرحلهٔ به دست آوردن انرژی زوج فیلترهای فرکانسی و مکانی مانند مقالهٔ \آ\ است؛ با این تفاوت که در این مقاله از ۳۰ فیلتر میان‌گذر نوع \مل{Butterworth} مرتبه ۲ استفاده می‌شود و لیست آن‌ها در قالب یک جدول در متن مقاله آمده است.

در این روش پس از این که انرژی زوج فیلترهای فرکانسی و مکانی به دست آمد، انرژی فیلترهای مکانی جمع می‌شود تا به هر فیلتر فرکانسی یک انرژی نسبت داده شود.\پانوشت{عملگر انرژی (واریانس) خطی نیست و بنابراین عملگر میانی \مل{CSP} در خروجی اثر دارد.} سپس، انرژی‌های کل هر باند فرکانسی مانند روش مقالهٔ \آ\ به یک الگوریتم \مل{Feature Selection} داده می‌شود و ۲۰ باند فرکانسی انتخاب می‌شوند. در هر باند فرکانسی نیز ۲۰ مولفهٔ \مل{CSP} با بیشترین انرژی انتخاب می‌شود. سپس ماتریس کواریانس برای هر باند فرکانسی محاسبه می‌شود و خروجی داده می‌شود.

بدین ترتیب، در این روش به ازای هر \مل{Trial} یک خروجی با عمل ۲۰، با ابعاد ۲۸ در ۲۸ خواهیم داشت (ماتریس کواریانس ۲۰ در ۲۰ است و از هر طرف با ۴ درایه صفر پد می‌شود).

\زیرقسمت{پیش پردازش \پ}
پیش پردازش و نمایش داده در این مقاله ساده‌تر است. در این مقاله، ابتدا کل داده‌ها از یک فیلتر مرتبه ۲ نوع \مل{Butterworth} با باند گذر ۴ تا ۴۰ هرتز عبور می‌کنند. سپس با استفاده از روش \مل{ICA}\پانویس{Independent Component Analysis} روی کانال‌های \مل{EEG}، ۲۰ پایهٔ جدید با بیشترین اطلاعات استخراج می‌شوند. الگوریتم \مل{ICA} مورد استفاده ذکر نشده است.

سپس داده‌ها \مل{Downsample} می‌شوند تا تعداد کل سمپل‌های سیگنال‌ها به ۳۰۰ برسد. سپس این سیگنال‌های به دست آمده به شبکه عصبی داده می‌شوند. به عبارتی، در این پیش پردازش به ازای هر \مل{Trial}، ۲۰ کانال ۳۰۰ تایی به دست می‌آید.

بین سه مقاله مورد بررسی، این مقاله تنها مقاله‌ای است که از در ورودی شبکهٔ عصبی از نمایش زمانی\پانویس{Temporal representation} سیگنال استفاده نموده است.

\قسمت{مقایسه معماری شبکه عصبی مقالات}
\ورودی{cnn}

