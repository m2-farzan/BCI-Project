\قسمت{بحث دربارهٔ نتایج}
برداشت کلی که از نتایج به دست می‌آید این است که در طراحی شبکه‌های عصبی برای پردازش \مل{EEG}، انتخاب مراحل پیش‌پردازش و نمایش داده نقش بسیار مهمی، اگر نه مهم‌تر از معماری خود شبکهٔ عصبی، در کیفیت مدل دارد.

برای اظهار نظر قطعی دربارهٔ صحت نتایج مقالات، موانعی وجود دارد. هر سه مقاله، هم در توصیف مدل پیشنهاد شده و هم در توصیف روش ارزیابی مدل، نقاط مبهمی داشته‌اند که البته در مورد مقالهٔ \آ\ و تا حدودی مقالهٔ \ب، این نقاط مبهم کم بوده‌اند. همچنین، تمام تست‌های مقالهٔ \ب\ روی داده‌های انحصاری انجام شده است که قابل بازتولید نیست. اگر با چشم‌پوشی از این نقاط، تست‌های انجام شده در این گزارش نمایندهٔ مدل‌های توصیف شده در مقالات مرجع در نظر گرفته شوند، می‌توان گفت که ادعای مقالات در تست‌های انجام شده مشاهده نشده است؛ زیرا هیچ کدام از روش‌ها نتوانسته‌اند خروجی متوسط بهتر از جداکنندهٔ خطی تولید کنند. دلایل احتمالی این امر در بخش بعد بررسی شده است.

\قسمت{نقد و بررسی به کار گیری \مل{CNN}}
مدل‌های \مل{CNN} مشخصه‌های عملکردی یکتایی دارند. یکی از مهمترین آن‌ها، حساسیت کم این مدل‌ها نسبت به انتقال\پانویس{Translation} یک مشخصه در ورودی است. این یکی از مهم‌ترین عواملی است که \مل{CNN}ها را برای کاربردهای پردازش تصویر مناسب کرده است. اما آیا این امر در پردازش سیگنال‌های مغزی یک خاصیت مفید است؟

در صورتی که سیگنال ورودی شبکهٔ عصبی از نوع زمانی(\مل{Temporal}) باشد (مانند مقالهٔ \آ\ و \پ)، این خاصیت باعث می‌شود مدل به اطلاعات زمانی مطلق اهمیت کمتری بدهد. اما این که آیا این یک نکتهٔ مثبت است یا منفی می‌تواند جای بحث داشته باشد. برای مثال، سیگنال‌های زمانی دیتاست مورد استفاده در این گزارش، ساختار زمانی ثابتی دارند و طول زمانی سیگنال و زمان تحریک مشخص است. یک قضاوت تخصصی در زمینهٔ علوم اعصاب می‌تواند مشخص کند که آیا در چنین موردی، زمان مطلق وقوع سیگنال‌های مغزی اهمیت زیادی دارد یا خیر. در مقابل، نکتهٔ مهم دیگری وجود دارد و آن این است که در کاربردهای واقع گرایانه، ممکن است پردازش یک استریم داده مطلوب باشد که در آن قالب زمانی مطلق سیگنال ارزش زیادی ندارد. در چنین مواردی، شاید \مل{CNN} برتری خود را نشان دهد.

در مواردی که ورودی شبکهٔ عصبی در دامنهٔ غیر زمانی است (مانند مقالهٔ \ب)، این خاصیت شبه تقارن \مل{CNN} نسبت به جابجایی ورودی می‌تواند معانی مختلفی داشته باشد. برای مثال در یک ورودی با دامنهٔ فرکانسی، کم اثر بودن جابجایی می‌تواند به این معنا باشد که تکرار یک الگو با شیفت فرکانسی مختلف توسط شبکه تشخیص داده می‌شود. این پدیده می‌تواند خوب یا بد باشد که نیاز به پیگیری توسط کارشناسان علوم اعصاب دارد.

نکتهٔ دیگر این است که \مل{CNN} معمولا به حجم عظیمی از داده‌ها نیاز دارد تا عملکرد قدرتمندی داشته باشد. حجم دیتاست‌های موجود درباره امواج مغزی امروزه محدود است (در مقیاس \مل{CNN}). این امر باعث می‌شود در بنچمارک‌های حال حاضر، \مل{CNN} عملکرد خیره‌کننده‌ای نداشته باشد؛ همانطور که در این گزارش نیز این مورد دیده شد. اما نکتهٔ حائز اهمیت این است که احتمالا در سال‌های پیش رو، داده برداری امواج مغزی رایج‌تر و آسان‌تر خواهد شد و همچنین با گسترش مواردی نظیر پردازش ابری، امکان جمع آوری حجم زیاد نمونه‌های نوار مغزی وجود خواهد داشت. بنابراین می‌توان نتیجه گرفت که در آینده مدل‌های \مل{CNN} می‌توانند به رقیب جدی مدل‌های رایج فعلی تبدیل شوند و شاید در نهایت جای آن‌ها را بگیرند.


\قسمت{پیشنهادها}
همانطور که در بخش تحلیل نتایج نشان داده شد، مدلی مرکب، متشکل از پیش‌پردازش توصیف شده در مقالهٔ \آ\ و شبکهٔ عصبی تولید شده در مقالهٔ \پ عملکرد خوبی را از خود نشان می‌دهد و قابل بررسی بیشتر می‌باشد. شکل‌های \رجوع{میله‌ای} و \رجوع{رگ} امکان مقایسهٔ عملکرد این روش پیشنهادی را با سایر روش‌ها به دست می‌دهند.

